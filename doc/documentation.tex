% *======================================================================*
%  Cactus Thorn template for ThornGuide documentation
%  Author: Ian Kelley
%  Date: Sun Jun 02, 2002
%
%  Thorn documentation in the latex file doc/documentation.tex
%  will be included in ThornGuides built with the Cactus make system.
%  The scripts employed by the make system automatically include
%  pages about variables, parameters and scheduling parsed from the
%  relevant thorn CCL files.
%
%  This template contains guidelines which help to assure that your
%  documentation will be correctly added to ThornGuides. More
%  information is available in the Cactus UsersGuide.
%
%  Guidelines:
%   - Do not change anything before the line
%       % START CACTUS THORNGUIDE",
%     except for filling in the title, author, date, etc. fields.
%        - Each of these fields should only be on ONE line.
%        - Author names should be separated with a \\ or a comma.
%   - You can define your own macros, but they must appear after
%     the START CACTUS THORNGUIDE line, and must not redefine standard
%     latex commands.
%   - To avoid name clashes with other thorns, 'labels', 'citations',
%     'references', and 'image' names should conform to the following
%     convention:
%       ARRANGEMENT_THORN_LABEL
%     For example, an image wave.eps in the arrangement CactusWave and
%     thorn WaveToyC should be renamed to CactusWave_WaveToyC_wave.eps
%   - Graphics should only be included using the graphicx package.
%     More specifically, with the "\includegraphics" command.  Do
%     not specify any graphic file extensions in your .tex file. This
%     will allow us to create a PDF version of the ThornGuide
%     via pdflatex.
%   - References should be included with the latex "\bibitem" command.c
%   - Use \begin{abstract}...\end{abstract} instead of \abstract{...}
%   - Do not use \appendix, instead include any appendices you need as
%     standard sections.
%   - For the benefit of our Perl scripts, and for future extensions,
%     please use simple latex.
%
% *======================================================================*
%
% Example of including a graphic image:
%    \begin{figure}[ht]
% 	\begin{center}
%    	   \includegraphics[width=6cm]{MyArrangement_MyThorn_MyFigure}
% 	\end{center}
% 	\caption{Illustration of this and that}
% 	\label{MyArrangement_MyThorn_MyLabel}
%    \end{figure}
%
% Example of using a label:
%   \label{MyArrangement_MyThorn_MyLabel}
%
% Example of a citation:
%    \cite{MyArrangement_MyThorn_Author99}
%
% Example of including a reference
%   \bibitem{MyArrangement_MyThorn_Author99}
%   {J. Author, {\em The Title of the Book, Journal, or periodical}, 1 (1999),
%   1--16. {\tt http://www.nowhere.com/}}
%
% *======================================================================*

% If you are using CVS use this line to give version information


\documentclass{article}


% Use the Cactus ThornGuide style file
% (Automatically used from Cactus distribution, if you have a
%  thorn without the Cactus Flesh download this from the Cactus
%  homepage at www.cactuscode.org)
%\usepackage{../../../../doc/latex/cactus}
\usepackage{cactus}
\usepackage{latexsym}
\usepackage{amssymb}
\usepackage{amsfonts}
\usepackage{amsmath}


\newcommand{\alp}{\alpha}
\newcommand{\flrwsolver}{\texttt{FLRWSolver}}


\begin{document}

% The author of the documentation
\author{H. ~J. Macpherson \textless h.macpherson@damtp.cam.ac.uk \textgreater,\\
 P.~D. Lasky, \\
 D.~J. Price} 

% The title of the document (not necessarily the name of the Thorn)
\title{FLRWSolver}

% the date your document was last changed, if your document is in CVS,
% please use:
%    \date{$ $Date: 2009-09-17 15:39:33 -0500 (Thu, 17 Sep 2009) $ $}

\date{\today}

\maketitle

% Do not delete next line
% START CACTUS THORNGUIDE

% Add all definitions used in this documentation here
%   \def\mydef etc

\begin{abstract}
  This thorn provides cosmological initial conditions based on a Friedmann-Lemaitre-Robertson-Walker (FLRW) spacetime, with and without small perturbations. 
\end{abstract}


%---------------------
\section{Introduction}
\label{sec:intro}
%---------------------

FLRW spacetime is a homogeneous, isotropic, expanding solution to Einstein's equations. This solution is the basis for the current standard cosmological model; $\Lambda$CDM. Here we provide a thorn to give initial conditions for cosmology using the Einstein Toolkit. We provide pure-FLRW spacetime (no perturbations), and linearly-perturbed FLRW spacetime (for various kinds of perturbations). Currently this thorn implements only a linearly-accurate solution of the constraint equations, so there is a second-order error in the solution from the beginning of the simulation. This can be removed by implementing exact solutions to the constraint equations; which can be added in the future. 


%--------------------------
\section{Use of this thorn}
\label{sec:use}
%--------------------------

To use this thorn to provide initial data for the {\tt ADMBase} variables $\alpha$, $\beta$, $g_{ij}$ and $K_{ij}$, and {\tt HydroBase} variables $\rho$, $v^i$ activate the thorn and set the following parameters: 
\begin{itemize}
	\item \texttt{HydroBase::initial\_hydro} = ``flrw''
	\item \texttt{ADMBase::initial\_data} = ``flrw''
	\item \texttt{ADMBase::initial\_lapse} = ``flrw''
	\item \texttt{ADMBase::initial\_shift} = ``flrw''
	\item \texttt{ADMBase::initial\_dtlapse} = ``flrw''
	\item \texttt{ADMBase::initial\_dtshift} = ``zero''
\end{itemize}

Template parameter files for an FLRW and single-mode perturbation to FLRW spacetime are provided in the {\tt par/} directory. Specific test parameter files for the cases presented in Section~\ref{sec:tests} are located in the {\tt par/tests/} directory. 

This thorn was first presented in \cite{macpherson2017}, and used further in \cite{macpherson2019}. For these cases, and all tests presented in Section~\ref{sec:tests}, we use this thorn with {\tt GRHydro} (see Section~\ref{sec:note}) and {\tt EOS\_Omni}. At the time of writing, neither of these thorns can handle a pure dust description, i.e. $P=0$. Instead we use a polytropic EOS and ensure $P\ll\rho$ by setting the polytropic constant accordingly. In \cite{macpherson2017} this was shown to be sufficient to match a dust FLRW evolution. 




%------------------
\section{Choosing initial conditions}
\label{sec:eqn}
%------------------

\subsection{FLRW spacetime \& background for perturbations} \label{sec:FLRWinit}

The FLRW line element in conformal time, $\eta$, is given by
\begin{equation}
	ds^2 = a^2(\eta) \left( - d\eta^2 + \delta_{ij}dx^i dx^j \right)
\end{equation}
where $a(\eta)$ is the scale factor describing the size of the Universe at time $\eta$. To initialise this spacetime, the user simply needs to specify \texttt{FLRW\_perturb} = ``no''. 

\vspace{2mm}
{\bf HERE is what we want to change things to (?) param names will change, basic structure should be the same}
\vspace{2mm}

\flrwsolver currently assumes an EdS model, i.e. matter-dominated, flat initial conditions. The user chooses the initial value of (the dimensionless) $HL$, i.e. the ratio of the Hubble horizon $d_H(z)\equiv c/H(z)$ to the physical \emph{proper} length of the box $L$. This is set using the parameter \texttt{FLRW\_init\_HL}. Since \flrwsolver assumes the EdS model, it is also assumed that the size of the Hubble horizon at $z=0$ is equivalent to $d_{H_0}=c/H_0\approx 6.6{\rm Gpc}$. The initial value of $HL$ set by the user, along with the chosen $z_{\rm init}$ (controlled by \texttt{FLRW\_init\_z} {\bf TO BE ADDED}), therefore implies the physical length of the box $L$ at the initial time. When setting pure FLRW or single-mode perturbations to FLRW (see fSection~\ref{sec:singlemode_ics} below), the physical side of the box is not necessary for creating initial data and therefore is only relevant when physically interpreting the results. This means the assumption of $d_{H_0}$ \emph{does not affect} pure-FLRW or single-mode perturbation initial conditions. However, when setting a powerspectrum of initial data (see Section~\ref{sec:pspec_ics} below), \flrwsolver also requires the \emph{physical} (comoving) box size (i.e., the proper length $L$ in Mpc at $z=0$) as input to create initial data from the relevant section of the CAMB matter power spectrum. It is up to the user to ensure the values of initial $HL$ and comoving box size are consistent ({\bf though maybe we can put a check in somehow? or just make it so that $z_{\rm init}$ sets the physical box size?}).



Physical units can then be achieved using the length unit, $L_{unit}$, defined by
\begin{equation}
	L_{code} \times L_{unit} = L_{phys},
\end{equation}
(easily translated to a time unit using the convention $c=1$) which can be used to translate $H(z)$ at any time, in code units, back to a physical $H(z)$ in $km/s Mpc$. 

\vspace{2mm}
{\bf BELOW is what is currently incorporated, to be changed to the above...}
\vspace{2mm}
In addition, the initial value for the dimensionless product of the conformal Hubble parameter and the box length, $\mathcal{H}_* L$, is specified with \texttt{FLRW\_init\_HL}, from which the initial background density in code units is set via the Friedmann equations with $G=c=1$, and $k=\Lambda=0$. The initial value for the scale factor $a$ is controlled with \texttt{FLRW\_init\_a}, and the initial lapse by \texttt{FLRW\_lapse\_value}, both of which default to 1. 

These parameters all still need to be set to describe the FLRW background if choosing \texttt{FLRW\_perturb} = ``yes''.

\subsection{Linear perturbations to FLRW spacetime}

Including scalar only perturbations to the FLRW metric in the longitudinal gauge gives
\begin{equation}\label{eq:perturbed_metric}
	ds^2 = a^2(\eta) \left[ - \left(1 + 2\psi\right) d\eta^2 + \left(1 - 2\phi \right) \delta_{ij}dx^i dx^j \right],
\end{equation}
where $\phi,\psi$ coincide with the Bardeen potentials \cite{bardeen1980} in this gauge. Assuming $\phi,\psi\ll1$ allows us to solve Einstein's equations using linear perturbation theory, giving the system of equations \cite{macpherson2017,macpherson2019}
 \begin{subequations} \label{eqs:perturbed_einstein}
	\begin{align}
		\nabla^{2}\phi - 3 \mathcal{H}\left(\phi' + \mathcal{H} \phi\right) &= 4\pi  \bar{\rho}\,\delta a^{2}, \label{eq:einstein_1} \\ 
		\mathcal{H} \partial_{i}\phi + \partial_{i}\phi' &= -4\pi \bar{\rho} \,a^{2} \delta_{ij}v^{j}, \label{eq:einstein_2} \\ 
		\phi'' + 3\mathcal{H}\phi' &=0, \label{eq:einstein_3}
	\end{align}	
\end{subequations}
and $\phi=\psi$. The perturbed rest-mass density is $\rho = \bar{\rho} \left(1 + \delta \right)$, with $\bar{\rho}$ the background FLRW density. We have $v^i = \delta v^i$, since $\bar{v}^i = 0$ for FLRW. In the above, $\mathcal{H}\equiv a'/a$ is the conformal Hubble parameter, where $'\equiv \partial/\partial\eta$ and $\partial_i \equiv \partial/\partial x^i$. Solving the above system, choosing \emph{only} the growing mode (see \cite{macpherson2019} for more details), gives
\begin{subequations} \label{eqs:linear_solnsg0}
    \begin{align}
    	\phi &= f(x^{i}), \label{eq:linear_phi}\\
     	\delta &= \frac{a_{\mathrm{init}}}{4\pi\rho^{*}} \xi^{2}\, \nabla^{2}f(x^{i}) - 2 \,f(x^{i}), \\
     	v^{i} &= -\sqrt{\frac{a_{\mathrm{init}}}{6\pi\rho^{*}}} \xi\, \partial^{i}f(x^{i}),
    \end{align}
\end{subequations}
where $a_{\mathrm{init}} = a(\eta_{\rm init})$, with $\eta_{\rm init}$ the initial simulation time, $\rho^*\equiv \bar{\rho}a^3$ is the FLRW comoving (constant) density, and we use the scaled conformal time
\begin{equation}
	\xi \equiv 1 + \sqrt{\frac{2\pi\rho^{*}}{3\,a_\mathrm{init}}}\eta,
\end{equation}
for simplicity. In \eqref{eq:linear_phi} we have the freedom to choose the form of $f(x^i)$, so long as it has amplitude such that $\phi\ll1$. Equations \eqref{eqs:linear_solnsg0} denote the standard form of linear perturbations implemented in this thorn. See \cite{macpherson2017} and \cite{macpherson2019} for more details. 

In this thorn the initial scale factor $a_{\rm init}$ is set with \texttt{FLRW\_init\_a}, and the initial background FLRW density $\bar{\rho}_{\rm init}$ is set implicitly from \texttt{FLRW\_init\_HL}, which correspondingly sets $\rho^*$.

Below we outline several different choices for perturbations, and how to set these using the thorn parameters.

%----------------
\subsubsection{Single-mode perturbation}\label{sec:singlemode_ics}
%----------------

This initial condition sets $\phi$ as a sine-wave function, and the corresponding density and velocity perturbations set using \eqref{eqs:linear_solnsg0}. Choose \texttt{FLRW\_perturb\_type}=``single\_mode''. The parameter \texttt{FLRW\_perturb\_direction} controls in which spatial dimension to apply the perturbation, and will set either $\phi=f(x^1),f(x^2),f(x^3)$, or $f(x^i)$ depending on the choice. For example, choosing \texttt{FLRW\_perturb\_direction}=``all'' we have
\begin{equation}\label{eq:phi}
	\phi = \phi_{0} \sum_{i=1}^{3} \mathrm{sin}\left(\frac{2\pi x^{i}}{\lambda} - \theta \right),
\end{equation}
where $\lambda$ is the wavelength of the perturbation, $\theta$ is some phase offset, and $\phi_0\ll1$. This gives the density and velocity perturbation as, respectively, \cite{macpherson2017}
\begin{align} 
	\delta &= - \left[ \left(\frac{2\pi}{\lambda}\right)^{2} \frac{a_{\mathrm{init}}}{4\pi\rho^{*}} + 2\right] \phi_{0} \sum_{i=1}^{3} \mathrm{sin}\left(\frac{2\pi x^{i}}{\lambda} - \theta \right),\label{eq:initial_delta}\\
	v^{i} &= \frac{2\pi}{\lambda}\sqrt{\frac{a_{\mathrm{init}}}{6\pi\rho^{*}}}\, \phi_{0}\, \mathrm{cos}\left(\frac{2\pi x^{i}}{\lambda} - \theta \right). \label{eq:initial_deltav}
\end{align}
The wavelength, $\lambda$, of the perturbation is controlled by \texttt{single\_perturb\_wavelength}, given as a fraction of the total box length (to ensure periodicity is satisfied). The phase offset $\theta$ is controlled by {\tt phi\_phase\_offset}, and the amplitude is set by \texttt{phi\_amplitude}, which must be set such that $\phi_0\ll1$, and $\lambda, \rho^*$ must also be chosen such that the corresponding density and velocity perturbations are also small enough to satisfy the linear approximation. Set the FLRW background according to Section~\ref{sec:FLRWinit}.

%----------------
\subsubsection{Power spectrum of perturbations}\label{sec:pspec_ics}
%----------------

We can instead choose the initial conditions to be a power spectrum of fluctuations, to better mimic the early state of the Universe. See \cite{macpherson2019} for more details. 
Note these fluctuations are still assumed to be linear perturbations to an FLRW background. 

This is implemented by reading in a two-column text file with the 3-dimensional matter power spectrum $P(k)$, where the wavenumber is $k=\sqrt{k_x^2+k_y^2+k_z^2}$. It is assumed the text file is the same format as the matter power spectrum output from CAMB \cite{lewis2002}, i.e. a text file with two columns $k, P(k)$ (in units of  and a one-line comment at the top, with \emph{no} line for $k=0$. Any file in this format will work, in units corresponding to CAMB output\footnote{see https://camb.info}.

Using a slightly modified version of the \texttt{c2raytools}\footnote{https://github.com/hjens/c2raytools} Python package, \texttt{FLRWSolver} generates a Gaussian random field that follows the given power spectrum. This is the density perturbation to the FLRW background. Assuming the metric \eqref{eq:perturbed_metric}, we use equations \eqref{eqs:linear_solnsg0} to find the corresponding velocity and metric perturbations for this density perturbation in Fourier space.

This initial condition is chosen by setting \texttt{FLRW\_perturb\_type}=``powerspectrum''. Set the FLRW background according to Section~\ref{sec:FLRWinit}, and set the path to (and name of) the text file containing the matter power spectrum using \texttt{FLRW\_powerspectrum\_file}. The Gaussian random field will be drawn with seed corresponding to \texttt{FLRW\_random\_seed}, so keep this constant to draw the same field more than once. You also must specify the size of the domain in comoving Mpc (cMpc) using \texttt{FLRW\_boxlength}, as physical units are required to generate the dimensionless perturbations from the power spectrum. 


%----------------
\subsubsection{Linear perturbations from user-specified files}
%----------------

There is also a possibility for you to set up your own initial conditions and read them in to \texttt{FLRWSolver}. These must be in ascii format, and the 3D data must be stored in a specific way (this may be generalised to a more widely-used format at some point in the future). The data is assumed to be stored in 2D $N\times N$ slices of the $x-y$ plane, with the $z$ index increasing every $N$ rows, where $N=n+n_{\rm gh}$ is the resolution ($n$) including the number of ghost cells in each dimension ($n_{\rm gh}$; i.e., with $n_{\rm gh}/2$ ghost cells on each side). Therefore, the columns are the $x$ indices, and the rows are the $y$ indices, repeating every $N$ rows. The files are therefore assumed to include ghost cells, although these may be set to zero, since the ET asserts periodic boundary conditions.

The files are assumed to be linear perturbations to an FLRW spacetime in the longitudinal gauge, as with other forms of initial conditions. The parameter \texttt{FLRW\_deltafile} specifies the fractional density perturbation $\delta$, \texttt{FLRW\_phifile} for the metric perturbation $\phi/c^2$, and \texttt{FLRW\_velxfile}, \texttt{FLRW\_velyfile}, and \texttt{FLRW\_velzfile} for each component of the contravariant fluid three-velocity with respect to the Eulerian observer, $v^i/c$ (see \texttt{HydroBase} documentation for more details). All quantities are assumed to be specified in code units (normalised by the speed of light, $c$, with $\delta$ already dimensionless).

This option is useful for, e.g., running test simulations with a controlled number of modes (but keeping the gradients the same between resolutions) that can be created using \texttt{tools/make\_test\_ics.py} alongside \texttt{tools/cut\_powerspectrum.py}.


%%----------------
%\subsection{Exact perturbation of metric}
%%----------------
%
%{\bf NOTE: not yet implemented in FLRWSolver. Just a note here for now.}
%
%Instead of assuming linear perturbations (i.e. $\phi,\psi,\delta,\delta v^i \ll 1$), we can set a perturbation to the metric in the same way as \eqref{eq:perturbed_metric} and solve the Hamiltonian and momentum constraints exactly. The corresponding extrinsic curvature is
%\begin{equation}
%	K_{ij} = \frac{-a\dot{a}(1-2\phi)}{\alp} \delta_{ij},
%\end{equation}
%with trace
%\begin{equation}
%	K = - \frac{3\dot{a}}{a\alp}.
%\end{equation}
%
%Using the Mathematica package Riemannian Geometry and Tensor Calculus (RGTC; add cite), we calculate the Ricci three-curvature scalar ${}^3 R$, and solve the Hamiltonian constraint
%\begin{equation}
%	{}^3 R + K^2 - K_{ij}K^{ij} = 16\pi\rho
%\end{equation}
%to get the ADM density $\rho \equiv T_{\mu\nu} n^\mu n^\nu$. We then solve the momentum constraint
%\begin{equation}
%	D_i \left( K^{ij} - \gamma^{ij} K \right) = 8\pi S^i
%\end{equation}
%to find the momentum density $S^i$. The rest-frame density $\rho_0$ can be calculated from the ADM density and the momentum density using
%\begin{equation} \label{eq:rho0_fromADM}
%	\rho_0 = \rho - \frac{S^i S_i}{\rho}.
%\end{equation}
%The momentum density is
%\begin{align}
%	S^i &= - n_\mu T^{i\mu}, \label{eq:Si}\\
%	\Rightarrow S^i S_i &= \rho_0 \Gamma^2 u^i u_i,
%\end{align}
%and the ADM density can be written using its definition as
%\begin{equation}
%	\rho = \rho_0 \Gamma^2,
%\end{equation}
%and so \eqref{eq:rho0_fromADM} becomes
%\begin{equation}
%	\rho - \frac{S^i S_i}{\rho} = \rho_0 \Gamma^2 - u^i u_i.
%\end{equation}
%Now using the normalisation condition 
%\begin{align}
%	u_\mu u^\mu &= -1, \\
%	u_0 u^0 + u_i u^i &= -1, \\
%	\Rightarrow u_i u^i &= \Gamma^2 - 1.
%\end{align}
%\textbf{This somehow gives $\rho_0$. Need to confirm how...} We find
%\begin{equation}
%	\rho_0 = \frac{1}{16\pi} \rho_{\rm curv} - \frac{\dot{a}^2 \partial_i (\alp) \partial^i (\alp)}{\pi a^4 \alp^4 (1-2\phi) \rho_{\rm curv}}, 
%\end{equation}
%where, for convenience we have defined
%\begin{align}
%	\rho_{\rm curv} &= \frac{6\dot{a}^2}{a^2 \alp^2} - \frac{2a^4}{\gamma} \left[ (4\phi - 2) \nabla^2 \phi - 3\partial_i (\phi) \partial^i (\phi) \right], \\
%	&= \frac{6\dot{a}^2}{a^2 \alp^2} + {}^3 R,
%\end{align}
%and $\gamma = a^6 (1-2\phi)^3$ is the determinant of the spatial metric. The fluid three-velocity w.r.t an Eulerian observer, $v^i$, can also be constructed from the momentum density and ADM density as follows, using \eqref{eq:Si},
%\begin{align}
%	\frac{S^i}{\rho} &= \frac{\alp T^{i0}}{\rho_0\Gamma^2}, \\
%	&= \frac{\rho_0 \Gamma^2 v^i}{\rho_0\Gamma^2}, \\
%	&= v^i.
%\end{align}
%Using this, we find
%\begin{equation}
%	v^i = \frac{- 4\dot{a}\partial^i (\alp)}{a^3 (1-2\phi) \alp^2 \rho_{\rm curv}}.
%\end{equation}
%
%
%
%
%%----------------
%\subsection{More perturbations}
%%----------------
%Several test cases for initial conditions are also included in the current up-to-date version of this thorn, however since these are just tests and not intended for public use; they are not detailed here. These are in the files {\tt FLRW\_SynchComoving.F90} and {\tt FLRW\_FramedragTest.F90}. Don't worry 'bout it.
%


%--------------------------
\section{Gauge}
\label{sec:gauge}
%--------------------------

The initial value of the lapse, $\alpha$ is set using \texttt{FLRW\_lapse\_value}. In a perturbed FLRW spacetime (using {\tt single\_mode} or {\tt powerspectrum}) this is the background value of the lapse, which is perturbed according to \eqref{eq:perturbed_metric}. This thorn uses $\beta^i=0$ always. Evolution of lapse is set as usual using the chosen evolution thorn. See \cite{macpherson2019} for a discussion about appropriate gauges for nonlinear structure formation. 




%--------------------------
\section{Notes on using {\tt GRHydro} for cosmology}
\label{sec:note}
%--------------------------

The work done in \cite{macpherson2017} and \cite{macpherson2019} using this thorn used {\tt GRHydro} for the hydrodynamic evolution. Below we offer a few notes and suggestions on using this thorn for cosmological evolutions. 

%----------------
\subsection{Equation of state}
%----------------

In deriving the initial conditions above, we assume a dust fluid in an FLRW spacetime, i.e. $P=0$. If using {\tt GRHydro} for the hydrodynamical evolution, it should be noted that this thorn does not work with an identically zero pressure. To compensate this, it is recommended to use a polytropic EOS, $P=K \rho^{\gamma}$ such that $P\ll\rho$. In \cite{macpherson2017} this is shown to be sufficient to match the evolution of a dust FLRW spacetime. Note that the polytropic constant $K$ controlled by \texttt{EOS\_Omni::poly\_k} will need to be adjusted accordingly for different choices of $\mathcal{H}_* L$ to ensure $P\ll\rho$.



%----------------
\subsection{The atmosphere}
%----------------

In {\tt GRHydro}, the atmosphere is used to solve the problem that the majority of the computational domain is essentially vacuum (when simulating compact objects; see the documentation). For cosmology, we do not need an atmosphere as the matter fluid is continuous across the whole domain (in the absence of shell-crossings). {\tt GRHydro} decides whether the position on the grid coincides with the atmosphere by checking several conditions. In some cases we have found that regions of the domain are flagged as being in ``the atmosphere'' for our cosmological simulations, whereas this is realistically not necessary. This causes these regions in the domain to be set automatically to the value of {\tt rho\_abs\_min}. To avoid this, we have located the particular line that is causing this behaviour and commented it out. This will have no effect on the evolution as we know the purpose of the atmosphere is for simulations of compact objects, and not cosmology. If using {\tt GRHydro} {\bf exclusively} for cosmology, in the code {\tt GRHydro\_UpdateMask.F90} comment out line 76 that sets {\tt atmosphere\_mask\_real(i,j,k) = 1}. If using {\tt GRHydro} for any simulations of compact objects, be sure to change this back to the distributed form, as in these cases the atmosphere is required. 





% --------------------------------------------
\section{Examples of initial conditions} \label{sec:tests}
% --------------------------------------------

Here we present some example initial data using this thorn with various parameter choices. The exact parameter files used to initialise each of the cases below are located in the \texttt{par/test/} directory. 

%%----------------------------------------
%\subsection{FLRW Spacetime}
%%----------------------------------------
%
%\emph{Is there anything to be said about this IC? Maybe something about the background spacetime and interpretation? Maybe fill this in when you have better user settings for the background.}
%

%----------------------------------------
\subsection{Single-mode linear perturbations}
%----------------------------------------

Here we consider only initial perturbation containing a single-wavelength mode. These are initialised first by choosing \texttt{FLRWSolver::perturb} =``yes'' and \texttt{FLRWSolver::perturb\_type}=``single\_mode''. 

For these tests we use \texttt{FLRWSolver::init\_HL}=2 unless otherwise stated. Any value can be chosen in practise, and for the last test we demonstrate this is possible by choosing \texttt{FLRWSolver::init\_HL}=10. We emphasise that care must be taken to ensure consistent values of $\phi_0$ and $\rho^*_{\rm init}$ are chosen such that the linear assumption of $\delta, v^i \ll 1$ is still satisfied. This can be done using \eqref{eq:initial_delta} and \eqref{eq:initial_deltav} along with the chosen wavelength of the perturbation. This thorn will give a warning if any of the perturbations are $\gtrsim 10^{-5}$. We also emphasise again that the polytropic constant \texttt{EOS\_Omni::poly\_k} must be chosen to be small enough such that $P \ll \rho$ if using \texttt{GRHydro} and a polytropic EOS. Note that currently no other evolution thorns or EOS choices have been tested with this thorn. For these tests we use \texttt{EOS\_Omni::poly\_k}=$10^{-5}$ and \texttt{EOS\_Omni::poly\_gamma}=$2$ unless otherwise stated.


%--------------------
\subsubsection{Test 1: FLRW\_singlemode\_AllDir\_L1\_phi1e-8} \label{sec:test1}
%--------------------

For this test we consider a perturbation similar to that considered in \cite{macpherson2017}. The relevant parameters are:

\begin{itemize}
	\item \texttt{FLRWSolver::perturb\_direction}=``all''
	\item \texttt{FLRWSolver::single\_perturb\_wavelength}=1
	\item \texttt{FLRWSolver::phi\_amplitude} = 1e-8
\end{itemize}
This means we will see a perturbation in all $x,y$, and $z$ directions as per \eqref{eq:phi} with wavelength equal to the box length, and the amplitude of the metric perturbation $\phi_0$ is $10^{-8}$. The resulting initial conditions are shown in Figure~\ref{fig:test1}. 

\begin{figure}[h!]
	\begin{center}
	   \includegraphics[width=\textwidth]{/Users/hayleymac/simulations/notebooks/FLRWSolver_Tests_FLRW_singlemode_AllDir_L1_phi1e-8.pdf}
	\end{center}
	\caption{Top left to bottom right: lapse, metric component $g_{xx}$, extrinsic curvature component $K_{xx}$, rest-mass density, and velocity component $v^x$. Here we show an $x-y$ slice through the mid-plane of the Cactus HDF5 output data (which includes ghost cells) for the test described in Section~\ref{sec:test1}.}
	\label{fig:test1}
\end{figure}



%--------------------
\subsubsection{Test 2: FLRW\_singlemode\_AllDir\_L1\_phi1e-8\_offset} \label{sec:test2}
%--------------------

For this test the relevant parameters we are considering are:

\begin{itemize}
	\item \texttt{FLRWSolver::perturb\_direction}=``all''
	\item \texttt{FLRWSolver::single\_perturb\_wavelength}=1.0
	\item \texttt{FLRWSolver::phi\_phase\_offset}=1.5
	\item \texttt{FLRWSolver::phi\_amplitude} = 1e-8
\end{itemize}
This means we will see a perturbation in all $x,y$, and $z$ directions as per \eqref{eq:phi} with wavelength equal to the box length and a phase offset of $\theta=1.5$. The amplitude of the metric perturbation $\phi_0$ is $10^{-8}$. The resulting initial conditions are shown in Figure~\ref{fig:test2}.

\begin{figure}[ht]
	\begin{center}
	   \includegraphics[width=\textwidth]{/Users/hayleymac/simulations/notebooks/FLRWSolver_Tests_FLRW_singlemode_AllDir_L1_phi1e-8_offset.pdf}
	\end{center}
	\caption{Top left to bottom right: lapse, metric component $g_{xx}$, extrinsic curvature component $K_{xx}$, rest-mass density, and velocity component $v^x$. Here we show an $x-y$ slice through the mid-plane of the Cactus HDF5 output data (which includes ghost cells) for the test described in Section~\ref{sec:test2}.}
	\label{fig:test2}
\end{figure}


%--------------------
\subsubsection{Test 3: FLRW\_singlemode\_AllDir\_L05\_phi1e-8} \label{sec:test3}
%--------------------

For this test the relevant parameters we are considering are:

\begin{itemize}
	\item \texttt{FLRWSolver::perturb\_direction}=``all''
	\item \texttt{FLRWSolver::single\_perturb\_wavelength}=0.5
	\item \texttt{FLRWSolver::phi\_amplitude} = 1e-8
\end{itemize}
This means we will see a perturbation in all $x,y$, and $z$ directions as per \eqref{eq:phi} with wavelength equal to half of the box length, and the amplitude of the metric perturbation $\phi_0$ is $10^{-8}$. The resulting initial conditions are shown in Figure~\ref{fig:test3}.

\begin{figure}[ht]
	\begin{center}
	   \includegraphics[width=\textwidth]{/Users/hayleymac/simulations/notebooks/FLRWSolver_Tests_FLRW_singlemode_AllDir_L05_phi1e-8.pdf}
	\end{center}
	\caption{Top left to bottom right: lapse, metric component $g_{xx}$, extrinsic curvature component $K_{xx}$, rest-mass density, and velocity component $v^x$. Here we show an $x-y$ slice through the mid-plane of the Cactus HDF5 output data (which includes ghost cells) for the test described in Section~\ref{sec:test3}.}
	\label{fig:test3}
\end{figure}


%--------------------
\subsubsection{Test 4: FLRW\_singlemode\_xDir\_L1\_phi1e-8} \label{sec:test4}
%--------------------

For this test the relevant parameters we are considering are:

\begin{itemize}
	\item \texttt{FLRWSolver::perturb\_direction}=``x''
	\item \texttt{FLRWSolver::single\_perturb\_wavelength}=1.0
	\item \texttt{FLRWSolver::phi\_amplitude} = 1e-8
\end{itemize}
This means we will see a perturbation in the $x$ direction only, with wavelength equal to the box length, and the amplitude of the metric perturbation $\phi_0$ is $10^{-8}$. The resulting initial conditions are shown in Figure~\ref{fig:test4}. These initial conditions are trivially extended to other directions by choosing \texttt{FLRWSolver::perturb\_direction}=``y'' or ``z''. 

\begin{figure}[ht]
	\begin{center}
	   \includegraphics[width=\textwidth]{/Users/hayleymac/simulations/notebooks/FLRWSolver_Tests_FLRW_singlemode_xDir_L1_phi1e-8.pdf}
	\end{center}
	\caption{Top left to bottom right: lapse, metric component $g_{xx}$, extrinsic curvature component $K_{xx}$, rest-mass density, and velocity component $v^x$. Here we show an $x-y$ slice through the mid-plane of the Cactus HDF5 output data (which includes ghost cells) for the test described in Section~\ref{sec:test4}.}
	\label{fig:test4}
\end{figure}



%--------------------
\subsubsection{Test 5: FLRW\_singlemode\_AllDir\_L1\_phi1e-5\_HL10} \label{sec:test5}
%--------------------

For this test we demonstrate the ability to initialise with a different background density value, so long as the EOS and perturbation amplitudes are set accordingly. The relevant parameters are:

\begin{itemize}
	\item \texttt{FLRWSolver::perturb\_direction}=``all''
	\item \texttt{FLRWSolver::single\_perturb\_wavelength}=1
	\item \texttt{FLRWSolver::phi\_amplitude} = 1e-5
	\item \texttt{FLRWSolver::FLRW\_init\_HL} = 10
	\item \texttt{EOS\_Omni::poly\_k} = 1e-15
\end{itemize}
This means we will see a perturbation in all $x,y$, and $z$ directions as per \eqref{eq:phi} with wavelength equal to the box length, and the amplitude of the metric perturbation $\phi_0$ is $10^{-8}$. The resulting initial conditions are shown in Figure~\ref{fig:test5}. 

\begin{figure}[ht]
	\begin{center}
	   \includegraphics[width=\textwidth]{/Users/hayleymac/simulations/notebooks/FLRWSolver_Tests_FLRW_singlemode_AllDir_L1_phi1e-5_largeHL.pdf}
	\end{center}
	\caption{Top left to bottom right: lapse, metric component $g_{xx}$, extrinsic curvature component $K_{xx}$, rest-mass density, and velocity component $v^x$. Here we show an $x-y$ slice through the mid-plane of the Cactus HDF5 output data (which includes ghost cells) for the test described in Section~\ref{sec:test5}.}
	\label{fig:test5}
\end{figure}




%----------------------------------------
\subsection{Multi-mode linear perturbations}
%----------------------------------------

Here we show an example of initial conditions considering a power spectrum of density perturbations. Specifically, we show a low-resolution example of the initial conditions for the simulation presented in \cite{macpherson2019}. The generation of the initial conditions themselves will be included in this thorn in future, at which point the parameters chosen will differ. This will be updated here. 

%--------------------
\subsubsection{Test 6: FLRW\_32c\_1Gpc\_fullPk} \label{sec:test6}
%--------------------

For this test the relevant parameters we are considering are:

\begin{itemize}
	\item \texttt{FLRWSolver::perturb}=``yes''
	\item \texttt{FLRWSolver::perturb\_type}=``powerspectrum''
\end{itemize}
Here we are considering a $32^3$ domain with 1 Gpc$^3$ box length. The minimum scale sampled is $\lambda_{\rm min} = 2 \Delta x$, i.e. the full power spectrum to the Nyquist frequency is sampled. The power spectrum is output from CAMB\footnote{https://camb.info} at redshift $z=1100$. The resulting initial conditions are shown in Figure~\ref{fig:test6}.

\begin{figure}[ht]
	\begin{center}
	   \includegraphics[width=\textwidth]{/Users/hayleymac/simulations/notebooks/FLRWSolver_Tests_FLRW_32c_1Gpc_fullPk.pdf}
	\end{center}
	\caption{Top left to bottom right: lapse, metric component $g_{xx}$, extrinsic curvature component $K_{xx}$, rest-mass density, and velocity component $v^x$. Here we show an $x-y$ slice through the mid-plane of the Cactus HDF5 output data (which includes ghost cells) for the test described in Section~\ref{sec:test6}.}
	\label{fig:test6}
\end{figure}



% ------------------------------------------------------------------------------------------------------------------------------
% ------------------------------------------------------------------------------------------------------------------------------
\begin{thebibliography}{9}

\bibitem{bardeen1980} J. M. Bardeen, Phys. Rev. D 22, 1882 (1980)

\bibitem{macpherson2017} H. J. Macpherson, P. D. Lasky, and D. J. Price, Phys. Rev. D 95, 064028 (2017), arXiv:1611.05447

\bibitem{macpherson2019} H. J. Macpherson, D. J. Price, and P. D. Lasky, Phys. Rev. D99, 063522 (2019), arXiv:1807.01711

\end{thebibliography}
% ------------------------------------------------------------------------------------------------------------------------------
% ------------------------------------------------------------------------------------------------------------------------------


\include{interface}
\include{param}
\include{schedule}

% Do not delete next line
% END CACTUS THORNGUIDE


\end{document}
