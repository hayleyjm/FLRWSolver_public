% *======================================================================*
%  Cactus Thorn template for ThornGuide documentation
%  Author: Ian Kelley
%  Date: Sun Jun 02, 2002
%
%  Thorn documentation in the latex file doc/documentation.tex
%  will be included in ThornGuides built with the Cactus make system.
%  The scripts employed by the make system automatically include
%  pages about variables, parameters and scheduling parsed from the
%  relevant thorn CCL files.
%
%  This template contains guidelines which help to assure that your
%  documentation will be correctly added to ThornGuides. More
%  information is available in the Cactus UsersGuide.
%
%  Guidelines:
%   - Do not change anything before the line
%       % START CACTUS THORNGUIDE",
%     except for filling in the title, author, date, etc. fields.
%        - Each of these fields should only be on ONE line.
%        - Author names should be separated with a \\ or a comma.
%   - You can define your own macros, but they must appear after
%     the START CACTUS THORNGUIDE line, and must not redefine standard
%     latex commands.
%   - To avoid name clashes with other thorns, 'labels', 'citations',
%     'references', and 'image' names should conform to the following
%     convention:
%       ARRANGEMENT_THORN_LABEL
%     For example, an image wave.eps in the arrangement CactusWave and
%     thorn WaveToyC should be renamed to CactusWave_WaveToyC_wave.eps
%   - Graphics should only be included using the graphicx package.
%     More specifically, with the "\includegraphics" command.  Do
%     not specify any graphic file extensions in your .tex file. This
%     will allow us to create a PDF version of the ThornGuide
%     via pdflatex.
%   - References should be included with the latex "\bibitem" command.c
%   - Use \begin{abstract}...\end{abstract} instead of \abstract{...}
%   - Do not use \appendix, instead include any appendices you need as
%     standard sections.
%   - For the benefit of our Perl scripts, and for future extensions,
%     please use simple latex.
%
% *======================================================================*
%
% Example of including a graphic image:
%    \begin{figure}[ht]
% 	\begin{center}
%    	   \includegraphics[width=6cm]{MyArrangement_MyThorn_MyFigure}
% 	\end{center}
% 	\caption{Illustration of this and that}
% 	\label{MyArrangement_MyThorn_MyLabel}
%    \end{figure}
%
% Example of using a label:
%   \label{MyArrangement_MyThorn_MyLabel}
%
% Example of a citation:
%    \cite{MyArrangement_MyThorn_Author99}
%
% Example of including a reference
%   \bibitem{MyArrangement_MyThorn_Author99}
%   {J. Author, {\em The Title of the Book, Journal, or periodical}, 1 (1999),
%   1--16. {\tt http://www.nowhere.com/}}
%
% *======================================================================*

% If you are using CVS use this line to give version information


\documentclass{article}


% Use the Cactus ThornGuide style file
% (Automatically used from Cactus distribution, if you have a
%  thorn without the Cactus Flesh download this from the Cactus
%  homepage at www.cactuscode.org)
%\usepackage{../../../../doc/latex/cactus}
\usepackage{cactus}
\usepackage{latexsym}
\usepackage{amssymb}
\usepackage{amsfonts}
\usepackage{amsmath}


\begin{document}

% The author of the documentation
\author{H. ~J. Macpherson \textless h.macpherson@damtp.cam.ac.uk \textgreater,\\
 P.~D. Lasky, \\
 D.~J. Price} 

% The title of the document (not necessarily the name of the Thorn)
\title{FLRWSolver}

% the date your document was last changed, if your document is in CVS,
% please use:
%    \date{$ $Date: 2009-09-17 15:39:33 -0500 (Thu, 17 Sep 2009) $ $}

\date{\today}

\maketitle

% Do not delete next line
% START CACTUS THORNGUIDE

% Add all definitions used in this documentation here
%   \def\mydef etc

\begin{abstract}
  This thorn provides cosmological initial conditions based on a Friedmann-Lemaitre-Robertson-Walker (FLRW) spacetime, with and without small perturbations. 
\end{abstract}


%---------------------
\section{Introduction}
\label{sec:intro}
%---------------------

FLRW spacetime is a homogeneous, isotropic, expanding solution to Einstein's equations. This solution is the basis for the current standard cosmological model; $\Lambda$CDM. Here we provide a thorn to give initial conditions for cosmology using the Einstein Toolkit. We provide pure-FLRW spacetime (no perturbations), and linearly-perturbed FLRW spacetime (for various kinds of perturbations). 


%--------------------------
\section{Use of this thorn}
\label{sec:use}
%--------------------------

To use this thorn to provide initial data for the {\tt ADMBase} variables $\alpha$, $\beta$, $g_{ij}$ and $K_{ij}$, and the {\tt HydroBase} variables $\rho$, $v^i$ just activate the thorn and set the parameters {\tt HydroBase::initial\_hydro, ADMBase::initial\_data, ADMBase::initial\_lapse, ADMBase::initial\_shift, ADMBase::initial\_dtlapse} to ``flrw''. Also set {\tt ADMBase::initial\_dtshift} to ``zero''. Template parameter files for an FLRW and single-mode perturbation to FLRW spacetime are provided in the {\tt par/} directory. 





%------------------
\section{Equations}
\label{sec:eqn}
%------------------

The FLRW line element in conformal time, $\eta$, is given by
\begin{equation}
	ds^2 = a^2(\eta) \left( - d\eta^2 + \delta_{ij}dx^i dx^j \right)
\end{equation}
where $a(\eta)$ is the scale factor describing the size of the Universe at time $\eta$. To initialise this spacetime, the user simply needs to specify \texttt{FLRW\_perturb} = ``no''.

Including scalar only perturbations to the FLRW metric in the longitudinal gauge gives
\begin{equation}\label{eq:perturbed_metric}
	ds^2 = a^2(\eta) \left[ - \left(1-2\psi\right) d\eta^2 + \left(1+2\phi \right) \delta_{ij}dx^i dx^j \right],
\end{equation}
where $\phi,\psi$ coincide with the Bardeen potentials \cite{bardeen1980} in this gauge. Assuming $\phi,\psi\ll1$ allows us to solve Einstein's equations using linear perturbation theory, giving the system of equations \cite{macpherson2017,macpherson2019}
 \begin{subequations} \label{eqs:perturbed_einstein}
	\begin{align}
		\nabla^{2}\phi - 3 \mathcal{H}\left(\phi' + \mathcal{H} \phi\right) &= 4\pi  \bar{\rho}\,\delta a^{2}, \label{eq:einstein_1} \\ 
		\mathcal{H} \partial_{i}\phi + \partial_{i}\phi' &= -4\pi \bar{\rho} \,a^{2} \delta_{ij}v^{j}, \label{eq:einstein_2} \\ 
		\phi'' + 3\mathcal{H}\phi' &=0, \label{eq:einstein_3}
	\end{align}	
\end{subequations}
and $\phi=\psi$. The perturbed rest-mass density is $\rho = \bar{\rho} \left(1 + \delta \right)$, with $\bar{\rho}$ the background FLRW density. We have $v^i = \delta v^i$, since $\bar{v}^i = 0$ for FLRW. In the above, $\mathcal{H}\equiv a'/a$ is the conformal Hubble parameter, where $'\equiv \partial/\partial\eta$ and $\partial_i \equiv \partial/\partial x^i$. Solving the above system gives
\begin{subequations} \label{eqs:linear_solnsg0}
    \begin{align}
    	\phi &= f(x^{i}), \label{eq:linear_phi}\\
     	\delta &= \frac{a_{\mathrm{init}}}{4\pi\rho^{*}} \xi^{2}\, \nabla^{2}f(x^{i}) - 2 \,f(x^{i}), \\
     	v^{i} &= -\sqrt{\frac{a_{\mathrm{init}}}{6\pi\rho^{*}}} \xi\, \partial^{i}f(x^{i}),
    \end{align}
\end{subequations}
where $a_{\mathrm{init}} = a(\eta_{\rm init})$, with $\eta_{\rm init}$ the initial simulation time, $\rho^*\equiv \bar{\rho}a^3$ is the FLRW comoving (constant) density, and we use the scaled conformal time
\begin{equation}
	\xi \equiv 1 + \sqrt{\frac{2\pi\rho^{*}}{3\,a_\mathrm{init}}}\eta,
\end{equation}
for simplicity. In \eqref{eq:linear_phi} we have the freedom to choose the form of $f(x^i)$, so long as it has amplitude such that $\phi\ll1$. Equations \eqref{eqs:linear_solnsg0} denote the standard form of linear perturbations implemented in this thorn. See \cite{macpherson2017} and \cite{macpherson2019} for more details. 

In this thorn we set $a_{\rm init}=1$, and the initial background FLRW density $\bar{\rho}_{\rm init}$ is controlled by the parameter \texttt{FLRW\_init\_rho}, which correspondingly sets $\rho^*$. The conformal Hubble parameter $\mathcal{H}$ is set according to the Friedmann equations.

Below we outline several different choices for perturbations, and how to set these using the thorn parameters.

%----------------
\subsection{Single-mode perturbation}
%----------------

This initial condition sets $\phi$ as a sine-wave function, and the corresponding density and velocity perturbations set using \eqref{eqs:linear_solnsg0}. Choose \texttt{FLRW\_perturb\_type}=``single\_mode''. The parameter \texttt{FLRW\_perturb\_direction} controls in which spatial dimension to apply the perturbation, and will set either $\phi=f(x^1),f(x^2),f(x^3)$, or $f(x^i)$ depending on the choice. For example, choosing \texttt{FLRW\_perturb\_direction}=``all'' we have
\begin{equation}
	\phi = \phi_{0} \sum_{i=1}^{3} \mathrm{sin}\left(\frac{2\pi x^{i}}{\lambda} - \theta \right),
\end{equation}
where $\lambda$ is the wavelength of the perturbation, $\theta$ is some phase offset, and $\phi_0\ll1$. This gives the density and velocity perturbation as, respectively, \cite{macpherson2017}
\begin{align}
	\delta &= - \left[ \left(\frac{2\pi}{\lambda}\right)^{2} \frac{a_{\mathrm{init}}}{4\pi\rho^{*}} + 2\right] \phi_{0} \sum_{i=1}^{3} \mathrm{sin}\left(\frac{2\pi x^{i}}{\lambda} - \theta \right),\label{eq:initial_delta}\\
	v^{i} &= \frac{2\pi}{\lambda} -\sqrt{\frac{a_{\mathrm{init}}}{6\pi\rho^{*}}}\, \phi_{0}\, \mathrm{cos}\left(\frac{2\pi x^{i}}{\lambda} - \theta \right). \label{eq:initial_deltav}
\end{align}
The wavelength, $\lambda$, of the perturbation is controlled by \texttt{single\_perturb\_wavelength}, given as a fraction of the total box length (to ensure periodicity is satisfied). The phase offset $\theta$ is controlled by {\tt phi\_phase\_offset}, and the amplitude is set by \texttt{phi\_perturb\_amplitude}, which must be set such that $\phi_0\ll1$ such that the corresponding density and velocity perturbations are valid solutions to Einstein's equations under the linear approximation. 

%----------------
\subsection{Power spectrum of perturbations}
%----------------

We can instead choose the initial conditions to be a power spectrum of fluctuations, to better mimic the early state of the Universe. See \cite{macpherson2019} for more details. Currently, this is implemented in this thorn by reading in files giving initial conditions for $\phi, \delta,$ and $v^i$ calculated using an external code using the power spectrum of matter fluctuations in the CMB. These files are not included as a part of this thorn yet, and the full generation of these initial conditions will be included in the future. Currently, this choice will work for any form of initial conditions read in from a file, so long as it is in the gauge given in \eqref{eq:perturbed_metric} and satisfies the system \eqref{eqs:linear_solnsg0}. 

Regardless, choose this initial condition by setting \texttt{FLRW\_perturb\_type}=``powerspectrum''. The location of the files specifying $\phi, \delta$, and $v^i$ (in files named {\tt phi.dat, delta.dat, vel1.dat, vel2.dat}, and {\tt vel3.dat}, respectively) is set using the string {\tt FLRW\_ICs\_dir}. Note: these files must be written in a specific way: as a series of $x-y$ planes for each value of $z$. This will be irrelevant once the initial condition generator is incorporated into this thorn. 




%----------------
\subsection{More perturbations}
%----------------
Several test cases for initial conditions are also included in the current up-to-date version of this thorn, however since these are just tests and not intended for public use; they are not detailed here. These are in the files {\tt FLRW\_SynchComoving.F90} and {\tt FLRW\_FramedragTest.F90}. Don't worry 'bout it.



%--------------------------
\section{Gauge}
\label{sec:gauge}
%--------------------------

The initial value of the lapse, $\alpha$ is set using \texttt{FLRW\_lapse\_value}. In a perturbed FLRW spacetime (using {\tt single\_mode} or {\tt powerspectrum}) this is the background value of the lapse, which is perturbed according to \eqref{eq:perturbed_metric}. This thorn uses $\beta^i=0$ always. Evolution of lapse is set as usual using the chosen evolution thorn. See \cite{macpherson2019} for details about appropriate gauges for nonlinear structure formation. 




%--------------------------
\section{Notes on using {\tt GRHydro} for cosmology}
\label{sec:note}
%--------------------------

The work done in \cite{macpherson2017} and \cite{macpherson2019} using this thorn used {\tt GRHydro} for the hydrodynamic evolution. Below we offer a few notes and suggestions on using this thorn for cosmological evolutions. 

%----------------
\subsection{Equation of state}
%----------------

In deriving the initial conditions above, we assume a dust fluid in an FLRW spacetime, i.e. $P=0$. If using {\tt GRHydro} for the hydrodynamical evolution, it should be noted that this thorn does not work with an identically zero pressure. To compensate this, it is recommended to use a polytropic EOS, $P=K \rho^{\gamma}$ such that $P\ll\rho$. In \cite{macpherson2017} this is shown to be sufficient to match the evolution of a dust FLRW spacetime. 



%----------------
\subsection{The atmosphere}
%----------------

In {\tt GRHydro}, the atmosphere is used to solve the problem that the majority of the computational domain is essentially vacuum (when simulating compact objects; see the documentation). For cosmology, we do not need an atmosphere as the matter fluid is continuous across the whole domain (in the absence of shell-crossings). {\tt GRHydro} decides whether the position on the grid coincides with the atmosphere by checking several conditions. In some cases we have found that regions of the domain are flagged as being in ``the atmosphere'' for our cosmological simulations, whereas this is realistically not necessary. This causes these regions in the domain to be set automatically to the value of {\tt rho\_abs\_min}. To avoid this, we have located the particular line that is causing this behaviour and commented it out. This will have no effect on the evolution as we know the purpose of the atmosphere is for simulations of compact objects, and not cosmology. If using {\tt GRHydro} {\bf exclusively} for cosmology, in the code {\tt GRHydro\_UpdateMask.F90} comment out line 76 that sets {\tt atmosphere\_mask\_real(i,j,k) = 1}. If using {\tt GRHydro} for any simulations of compact objects, be sure to change this back to the distributed form, as in these cases the atmosphere is required. 




\begin{thebibliography}{9}

\bibitem{bardeen1980} J. M. Bardeen, Phys. Rev. D 22, 1882 (1980)

\bibitem{macpherson2017} H. J. Macpherson, P. D. Lasky, and D. J. Price, Phys. Rev. D 95, 064028 (2017), arXiv:1611.05447

\bibitem{macpherson2019} H. J. Macpherson, D. J. Price, and P. D. Lasky, Phys. Rev. D99, 063522 (2019), arXiv:1807.01711



\end{thebibliography}

\include{interface}
\include{param}
\include{schedule}

% Do not delete next line
% END CACTUS THORNGUIDE


\end{document}
